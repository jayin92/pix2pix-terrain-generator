\documentclass[16pt]{extarticle}
\usepackage[utf8]{inputenc}
\usepackage[margin=2cm]{geometry}
%加這個就可以設定字體
\usepackage{fontspec}
%使用xeCJK,其他的還有CJK或是xCJK
\usepackage{xeCJK}
\usepackage{xCJKnumb}
\usepackage{setspace}
\usepackage{enumitem}
\usepackage{indentfirst}

\setCJKmainfont{新細明體}

%中文自動換行
\XeTeXlinebreaklocale "zh"

%文字的彈性間距
\XeTeXlinebreakskip = 0pt plus 1pt

%設定段落之間的距離

\setlength{\parindent}{2em}
\setlength{\parskip}{1em}
\renewcommand{\baselinestretch}{1.1}

\title{利用生成對抗網路(GAN)生成擬真的山脈地形}
\author{李杰穎、程品奕}
\date{中華民國109年2月4日}

\begin{document}

\maketitle

\section*{摘要}
使用Google Maps擷取真實的地形,並利用一個二維的矩陣儲存特定位置的高度。後使用Perlin Noise生成類似山脈的地形,但因為此地形未經過河流的侵蝕,所以離真正的山脈地形仍有段距離,故我們可以訓練一個神經網路優化下雨的大小及水對山脈的影響,盡可能的模擬出接近真實山脈的地形。我們還可以利用一個神經網路判斷一個山脈是否是由人工生成出來的。所以這邊就有兩個神經網路,互相優化,有點像GAN的概念。

我是中文測試
\section*{壹、研究動機}
\section*{貳、研究目的}
\section*{參、研究設備及器材}
\section*{肆、研究過程或方法}
\section*{伍、研究結果}
\section*{陸、討論}
\section*{柒、結論}
\section*{捌、參考資料及其他}

\end{document}
